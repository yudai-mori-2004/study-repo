% !TEX TS-program = xelatex
% !TEX encoding = UTF-8 Unicode

\documentclass[a4paper,11pt]{article}

% 日本語フォント
\usepackage{xeCJK}
\setCJKmainfont{IPAexGothic}

% 余白・段落
\usepackage[a4paper,margin=25mm]{geometry}
\usepackage{parskip}               % 段落間の余白
\setlength{\parindent}{0pt}        % 行頭インデントなし

% 箇条書き「・」
\usepackage{enumitem}
\setlist[itemize,1]{label=・, left=0pt, nosep}

% セクション間の余白を微調整
\usepackage{titlesec}

\usepackage{listings}
\usepackage{xcolor}

\lstset{
  language=Python,
  basicstyle=\ttfamily\small,
  breaklines=true,
  columns=fullflexible,
  inputencoding=utf8,
  keywordstyle=\color{blue},
  commentstyle=\color{gray},
  stringstyle=\color{teal},
  showstringspaces=false,
}
\titlespacing*{\subsection}{0pt}{9ex}{1ex}

\begin{document}

%―――――――――――――――――――――――――――――――――――――――
% タイトルブロック
\begin{center}
  {\LARGE \bfseries ウェアラブル端末と心理入力アプリデータを統合した\\感情予測アルゴリズムの開発}\\[6ex]
\end{center}

\hrule\vspace{1ex}
{\normalsize 2025-06-13 進捗報告 \hfill 森雄大}\\
\hrule\vspace{5ex}

\section*{1. はじめに}
% 背景
\subsection*{研究の背景}
過度な「怒り・イライラ・焦燥感・不安・抑うつ」は、放置すると心身の疾患や対人トラブル、生産性低下を招く。  
自他ともに感情変化を捉えにくいことが、早期ケアを難しくしている。

% 先行研究
\subsection*{先行研究}
\begin{itemize}
  \item 外部研究(在宅勤務者30名):  
        脳波・心電図・呼吸数・動作量と心理指標を同時取得し、  
        ネガティブ感情と生体信号の相関を示した。
  \item 中村教授による大規模データ計測:  
        上記先行研究の知見を踏まえ、対象人数・測定項目を大幅に拡充。  
        大規模データセットを新たに収集。
\end{itemize}

% 研究の立ち位置
\subsection*{私の研究の立ち位置と目的}
教授が収集した大規模データ  
(心拍数・呼吸数・活動量・皮膚電気反応など)を使わせていただき解析を行う。  
私が設定した目的は、  
ウェアラブル端末だけでリアルタイムに  
ネガティブ/ポジティブの感情を高精度予測できる  
アルゴリズムを構築し、セルフケアに役立つ仕組みを提供すること。

% 期待される成果
\subsection*{期待される成果}
\begin{itemize}
  \item 利用者が自分のメンタル状態を「見える化」できるセルフケアツール
  \item 家族・職場との円滑な感情共有による早期サポート
  \item 生理心理学とメンタルヘルス技術の発展への寄与
\end{itemize}

\newpage
\section*{2. データ収集と解析の概要}
%―――――――――――――――――――――――――――――――――――――――
% データ収集・解析の概要
\subsection*{データの概要}
提供いただいた生体計測データはすべてEmpatica製E4リストバンドによる計測データであった。各被験者には以下の4種類のデータファイルが用意されており、心理入力アプリで感情を入力する直前15分のデータが全入力分含まれていると思われる。

1. 運動データ(加速度)

2. 皮膚表面温度データ

3. 心拍数データ

4. 皮膚電位活動データ

併せて、感情状態の入力タイミングをまとめたExcelシートも提供されており、1被験者あたり約15日間、1日約6回(午前9時~午後8時)ほど入力が行われている。被験者数は資料上で約150~270名との記載あり。1被験者あたりの入力回数は概ね100回前後とみられる。


感情状態の定量化には、

A. 外山美樹氏による楽観・悲観性尺度アンケート(個人の心理傾向を測定)

B. ラッセルの円環モデルに基づく9×9グリッドでのリアルタイム入力

の2手法が想定されているが、今回いただいたデータに含まれていたのはBの入力データのみであった。個人特性を考慮した汎用モデル構築には、Aのアンケート結果の利用が必要と考えられるので、データが取得済みであればぜひ解析に使用させていただきたいです。


\subsection*{データの形式}
IDが2のデータを用いて、計測データの形式を確認した。
% HDF5 ファイル概要(表形式で見やすく)
\begin{table}[ht]
  \centering
  \begin{tabular}{|l|l|c|l|}
    \hline
    \textbf{File} & \textbf{Dataset} & \textbf{Shape} & \textbf{Dtype} \\ \hline
    \texttt{data\_2\_E4\_act.h5} & \texttt{E4} & (10, 28799) & float64 \\ \hline
    \texttt{data\_2\_E4\_eda.h5} & \texttt{E4} & (1, 3599)   & float64 \\ \hline
    \texttt{data\_2\_E4\_rri.h5} & \texttt{E4} & (1, 3600)   & float64 \\ \hline
    \texttt{data\_2\_E4\_temp.h5} & \texttt{E4} & (1, 3599)   & float64 \\ \hline
  \end{tabular}
  %\caption{サンプル HDF5 ファイルのデータ概要}
  %\label{tab:h5_overview}
\end{table}

例えば、皮膚表面温度データであるtempは、4Hzでデータ取得をしている。約3600個のデータがあるということは、15分間のデータである。
つまり、このデータは被験者1(S01)の2回目の感情入力タイミングの直前15分の皮膚表面温度の時系列データである。



\newpage
\subsection*{データの読み込み方法}
以下のようなコードでh5ファイルの読み込みを行う。

\vspace{5ex}\hrule\vspace{1ex}
\begin{lstlisting}[language=Python, caption={HDF5ファイル読み込み関数}, label={lst:list_h5}]
import os
import h5py

def list_and_preview_h5_files(directory="sample", preview_shape=(2, 5)):
    """
    指定ディレクトリ内のすべての .h5 ファイルを走査し、
    各データセットの形状と冒頭の一部データを表示します。
    """
    for fname in sorted(os.listdir(directory)):
        if not fname.endswith(".h5"):
            continue
        filepath = os.path.join(directory, fname)
        print(f"\n=== File: {fname} ===")
        with h5py.File(filepath, "r") as f:
            for dset_name, dset in f.items():
                data = dset[()]
                print(f"- Dataset '{dset_name}': shape={data.shape}, dtype={data.dtype}")

if __name__ == "__main__":
    list_and_preview_h5_files()
\end{lstlisting}
\vspace{1ex}\hrule\vspace{1ex}

\newpage
\subsection*{全データの詳細}
以下に、データの形状や値の単位などのメタデータを表形式でまとめる。
\begin{table}[ht]
  \centering
  \begin{tabular}{|l|l|c|l|}
    \hline
    \textbf{ファイル名} & \textbf{説明} & \textbf{データの長さ} & \textbf{単位} \\ \hline
    \texttt{data\_N\_E4\_act.h5} & \texttt{x軸方向の生加速度(32Hz*15min)} & 28799 & 不明 \\ \hline
    \texttt{} & \texttt{y軸方向の生加速度(32Hz*15min)} & 28799 & 不明 \\ \hline
    \texttt{} & \texttt{z軸方向の生加速度(32Hz*15min)} & 28799 & 不明 \\ \hline
    \texttt{} & \texttt{身体運動ベクトルの大きさ(32Hz*15min)} & 28799 & 不明 \\ \hline
    \texttt{} & \texttt{x軸方向の加速度の重力成分(32Hz*15min)} & 28799 & 不明 \\ \hline
    \texttt{} & \texttt{y軸方向の加速度の重力成分(32Hz*15min)} & 28799 & 不明 \\ \hline
    \texttt{} & \texttt{z軸方向の加速度の重力成分(32Hz*15min)} & 28799 & 不明 \\ \hline
    \texttt{} & \texttt{x軸方向の身体運動成分(32Hz*15min)} & 28799 & 不明 \\ \hline
    \texttt{} & \texttt{y軸方向の身体運動成分(32Hz*15min)} & 28799 & 不明 \\ \hline
    \texttt{} & \texttt{z軸方向の身体運動成分(32Hz*15min)} & 28799 & 不明 \\ \hline

    \texttt{data\_N\_E4\_eda.h5} & \texttt{感情計測直前の皮膚電位データ(4Hz*15min)} & (1, 3599)   & 不明 \\ \hline
    \texttt{data\_N\_E4\_rri.h5} & \texttt{感情計測直前の心拍間隔データ(4Hz*15min)} & (1, 3600)   & 秒 \\ \hline
    \texttt{data\_N\_E4\_temp.h5} & \texttt{感情計測直前の皮膚温度データ(4Hz*15min)} & (1, 3599)   & 摂氏温度 \\ \hline
  \end{tabular}
  %\caption{サンプル HDF5 ファイルのデータ概要}
  %\label{tab:h5_overview}
\end{table}


対応するタイムスタンプのデータは、「all-grid-data-extracted2025」から取得できる。
計測ファイル名が「data-N-E4-act.h5」であれば対応するタイムスタンプはシートのN行目を参照すればよい。

\subsection*{簡易解析}
全被験者のデータをもとに簡単な解析を行う予定だったが、実行環境への全計測データのコピーが丸一日以上かかるため間に合わなかった。次回のToDoとする。

\subsection*{次回のToDo}
\begin{itemize}
  \item データの傾向を理解するための簡単な解析(一日の計測頻度の分散等)
  \item どんな回帰による予測が良いか、データの性質を踏まえて考察
\end{itemize}



\end{document}
